\documentclass[•]{article}
%this is a comment line.

\usepackage[margin=1 in]{geometry}
\usepackage{graphicx}
\geometry{a4paper}

%These are for source code snippets in 
\usepackage{listings}
\usepackage{color}
\definecolor{dkgreen}{rgb}{0,0.6,0}
\definecolor{gray}{rgb}{0.5,0.5,0.5}
\definecolor{mauve}{rgb}{0.58,0,0.82}
\lstset{frame=tb,
  language=C++,
  aboveskip=3mm,
  belowskip=3mm,
  showstringspaces=false,
  columns=flexible,
  basicstyle={\small\ttfamily},
  numbers=none,
  numberstyle=\tiny\color{gray},
  keywordstyle=\color{blue},
  commentstyle=\color{dkgreen},
  stringstyle=\color{mauve},
  breaklines=true,
  breakatwhitespace=true,
  tabsize=3
}

\title{Detailed Documentation of Task Kernels for Xilinx Platform}
\author{SalehJG}

\begin{document}
\maketitle

\section{Intro}
This document is meant to be a detailed guide to understand the source code for each kernels and the challenges that are known for each scenario.
Most of material here has documented while Deep-Point was under development, so, they might not be updated as you would expect but they are intended to be up to date.

\subsection{Repository Information}
This document is a part of Deep-Point-FPGA project which is open-source. The Deep-Point-FGPA is a subset of Deep-Point focusing on implementing it on FPGAs.

\subsection{Programming}
\begin{itemize}
\item Xilinx Vivado 2018.2
\item Xilinx SDAccel 2018.2
\item Cent OS 7.4 X64
\end{itemize}

\pagebreak

\section{Concat2}
Concatenates two tensors of rank four into one over a dimension.
\subsection{Top Function}
\begin{lstlisting}
void task_concat(
		float* inputTn1,
	    float* inputTn2,
	    float* outputTn,

		unsigned int dimA0,
		unsigned int dimA1,
		unsigned int dimA2,
		unsigned int dimA3,

		unsigned int dimB0,
		unsigned int dimB1,
		unsigned int dimB2,
		unsigned int dimB3)
\end{lstlisting}

\subsection{Usage}
Cocat Over Dim: 3
\vspace{0.5cm}
\begin{table}[htbp] % put table at (h:here, t:top, b:bottom, p:seperated page)
\caption{Usage Instances and Tensor Shapes}
\label{tab:shapes_concat}
	\begin{center}
		\begin{tabular}{|r|c|c|c|c|} 
		\hline	
		Tensor & Dim0 & Dim1 & Dim2 & Dim3\\ 
		\hline	
		InputTn1 &
			5 &
			1024 &
			1, 20 &
			3, 64, 128, 192 \\ 
		\hline
		InputTn2 &
			5 & 
			1024 & 
			1, 20 & 
			3, 64, 128 \\
		\hline
		OutputTn &
			Dim0 & 
			Dim1 & 
			Dim2 & 
			DimA3+DimB3 \\
		\hline
		\end{tabular}
	\end{center}
\end{table}

\subsection{Fixed Shape Instances}
\begin{table}[htbp] % put table at (h:here, t:top, b:bottom, p:seperated page)
\caption{Fixed Shaped Instances of Kernel Template}
\label{tab:shapes_concat}
	\begin{center}
		\begin{tabular}{|r|c|} 
		\hline	
		  & Instance 1\\ 
		\hline	
		DimA0 &
			5 \\ 
		\hline
		DimA1 & 
			1024\\
		\hline
		DimB0 & 
			5\\
		\hline
		DimB1 & 
			1024\\
		\hline
		\end{tabular}
	\end{center}
\end{table}






\pagebreak





\section{Sqrt}
Takes second root of each element in the input tensor of any rank.

\subsection{Top Function}
\begin{lstlisting}
void task_sqrt(
        float* inputTn,
        float* outputTn,
        unsigned long len)
\end{lstlisting}

\subsection{Usage}
Not available.

\subsection{Fixed Shape Instances}
None.






\pagebreak






\section{ReduceMax}
A reduction kernel with \textit{max} operator reducing on the element over the given dimension. The kernel is designed to operate on tensors of \textbf{rank three} for simplicity but it takes tensors of \textbf{rank four} in the usage cases as it is described below.
\subsection{Top Function}
\begin{lstlisting}
void task_reducemax(
        float* inputTn,
        float* outputTn,
		const unsigned int dim0,
		const unsigned int dim1,
		const unsigned int dim2,
		const int overaxis0,
		const int overaxis1,
		const int overaxis2)
\end{lstlisting}

\subsection{Usage}
Combination: FTF
\vspace{0.5cm}
\begin{table}[htbp] % put table at (h:here, t:top, b:bottom, p:seperated page)
\caption{Usage Instances and Tensor Shapes}
	\begin{center}
		\begin{tabular}{|r|c|c|c|c|} 
		\hline	
		Tensor & Dim0 & Dim1 & Dim2 & Dim3\\ 
		\hline	
		InputTn with reduction dim of 1 &
			5 &
			1024 &
			1 &
			1024 \\ 
		\hline	
		InputTn with reduction dim of 2 &
			5 &
			1024 &
			20 &
			64, 128 \\ 
		\hline
		\end{tabular}
	\end{center}
\end{table}

\begin{table}[htbp] % put table at (h:here, t:top, b:bottom, p:seperated page)
\caption{Kernel inputs for given usage instances}
	\begin{center}
		\begin{tabular}{|r|c|c|c|c|} 
		\hline	
		Reduction Dim & Arg Dim0 & Arg Dim1 & Arg Dim2 & Condition\\ 
		\hline	
		1 &
			Dim0 &
			Dim1 &
			Dim3 &
			Dim2=1 \\ 
		\hline	
		2 &
			Dim0*Dim1 &
			Dim2 &
			Dim3 &
			None \\ 
		\hline
		\end{tabular}
	\end{center}
\end{table}

\subsection{Fixed Shape Instances}
None.






\pagebreak

\end{document}
